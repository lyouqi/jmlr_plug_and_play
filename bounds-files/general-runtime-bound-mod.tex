\begin{thm}
\label{thm:ct-runtime}
%For any dual-tree algorithm using cover trees and the standard cover tree
%dual-tree pruning traversal (Algorithm \ref{alg:cover-tree-dual}) with reference
%set $S_r$ that has expansion constant $c_r$ and query set $S_q$, the runtime is
%bounded above by $O(c_r^4 | R^* | \nu \chi \psi N)$, where $ | R^* | $ is
%the maximum size of the reference set $R$ (line
%\ref{alg:line:ct-dual-input}) at any point in the dual-tree pruning traversal,
%$\chi$ is the longest possible running time of \texttt{BaseCase()}, and $\psi$
%is the longest possible running time of \texttt{Score()}.
Given a reference set $S_r$ of size $N$ with an expansion constant $c_r$ and a
set of queries $S_q$ of size $O(N)$, a standard cover tree based dual-tree
algorithm (Algorithm \ref{alg:cover-tree-dual}) takes $O(c_r^4 | R^* | \chi
\psi (N + \theta))$, where $ | R^* | $ is the maximum size of the reference set $R$ (line
\ref{alg:line:ct-dual-input}) during the dual-tree recursion, $\chi$ is
the maximum possible runtime of \texttt{BaseCase()}, $\psi$ is the maximum
possible runtime of \texttt{Score()}, and $\theta$ is

\begin{equation}
\theta = \max(\log_2(\eta_r / \eta_q), 0) + b + N \max(\log_2(\delta_q / \delta_r), 0)
\end{equation}

\noindent with $\eta_q$ as the maximum pairwise distance between points in
$S_q$, $\delta_q$ as the minimum nonzero pairwise distance between points in
$S_q$, $\eta_r$ as the maximum pairwise distance between points in $S_r$,
$\delta_r$ as the minimum nonzero pairwise distance between points in $S_r$, and
$b$ as some type of measure of the imbalance of the tree which is defined as

\begin{equation}
b = \sum_{\mathscr{N}_q \in \mathscr{T}_q} \left(\sum_{\mathscr{N}_{qc} \in
\mathscr{C}(\mathscr{N}_q), s_{qc} \ne -\infty} s_q - s_{qc} - 1 \right) +
\left(\sum_{\mathscr{N}_{qc} \in \mathscr{C}(\mathscr{N}_q), s_{qc} = -\infty}
\max(s_q - s_q^{\min} - 1, 0) \right).
\end{equation}
\end{thm}

\vspace*{-1.23em}
\begin{proof}
First, split the algorithm into two parts: reference recursions (lines
\ref{alg:line:ct-dual-ref-recursion-start}--\ref{alg:line:ct-dual-ref-recursion-end})
and query recursions (lines
\ref{alg:line:ct-dual-query-recursion-start}--\ref{alg:line:ct-dual-query-recursion-end}).
The runtime of the algorithm is bounded as the runtime of a reference recursion
times the total number of reference recursions plus the total runtime of all
query recursions.

Consider a reference recursion (lines
\ref{alg:line:ct-dual-ref-recursion-start}--\ref{alg:line:ct-dual-ref-recursion-end}).
The work done in the base case loop from lines
\ref{alg:line:ct-dual-base-case-start}--\ref{alg:line:ct-dual-base-case-end} is
$O(\chi | R |)$.  Define $| R^* |$ to be the largest set $|R|$ for any scale
$s_r^{\max}$ and any query node $\mathscr{N}_q$ during the course of the
algorithm; then, it is true that $| R | \le | R^* |$.
%
Then, lines \ref{alg:line:ct-dual-ref-children} and
\ref{alg:line:ct-dual-ref-score} take $O(c_r^4 \psi | R |) \le O(c_r^4 \psi |
R^* |)$ time, because each reference node has $c_r^4$ children.  So, one
full reference recursion takes $O(c_r^4 \psi \chi | R^* |)$ time.

Now, note that there are $O(N)$ nodes in $\mathscr{T}_q$.  Thus, line
\ref{alg:line:ct-dual-query-recursion} is visited $O(N)$ times.  The amount of
work in line \ref{alg:line:ct-dual-query-pruning}, like in the reference
recursion, is bounded as $O(c_r^4 \psi | R^* |)$. Therefore, the total
runtime of all query recursions is $O(c_r^4 \psi | R^* | N)$.

Lastly, we must bound the total number of reference recursions.  Reference
recursions happen in three cases: \textit{(1)} $s_r^{\max}$ is greater than the
scale of the root of the query tree (no query recursions have happened yet);
\textit{(2)} $s_r^{\max}$ is less than or equal to the scale of the root of the
query tree, but is greater than the minimum scale of the query tree that is not
$-\infty$; \textit{(3)} $s_r^{\max}$ is less than the minimum scale of the query
tree that is not $-\infty$.

First, consider case \textit{(1)}.  If the scale of the root of the query tree
is $s_q^T$ and the scale of the root of the reference tree is $s_r^T$, then
$s_r^T - s_q^T$ recursions of this sort occur before the first query recursion.

Next, consider case \textit{(2)}.  Each query recursion implies at least one
reference recursion; thus there will be at least $O(N)$ recursions.  Any
extra recursions above and beyond this occur because the query node does not
have scale which is exactly one less than its parent.  When the query node is
not a leaf, the number of extra reference recursions is bounded by the
difference between the scales of the query node and its parent minus one (one is
subtracted to account for the one reference recursion that must happen).  Each
query node is visited once, and therefore the total number of extra recursions
when the query node is not a leaf is bounded by

\begin{equation}
\sum_{\mathscr{N}_q \in \mathscr{T}_q} \sum_{\mathscr{N}_{qc} \in
\mathscr{C}(\mathscr{N}_q), s_{qc} \ne -\infty} s_q - s_{qc} - 1.
\end{equation}

When the query node is a leaf, then the number of recursions in case
\textit{(2)} for that query node is bounded by the difference between the scale
of the query node's parent and the smallest scale of a non-leaf node in the
query tree (call this $s_q^{\min}$) minus one:

\begin{equation}
\sum_{\mathscr{N}_q \in \mathscr{T}_q} \sum_{\mathscr{N}_{qc} \in
\mathscr{C}(\mathscr{N}_q), s_{qc} = -\infty} \max(s_q - s_q^{\min} - 1, 0).
\end{equation}

The $\max$ is necessary to account for the case where the query node's parent
has scale $s_q^{\min}$; no reference recursions of type \textit{(2)} happen at
all in this case.  Therefore the total number of extra reference recursions of
type \textit{(2)} is bounded by $b$, where

\begin{equation}
b = \sum_{\mathscr{N}_q \in \mathscr{T}_q} \left(\sum_{\mathscr{N}_{qc} \in
\mathscr{C}(\mathscr{N}_q), s_{qc} \ne -\infty} s_q - s_{qc} - 1 \right) +
\left(\sum_{\mathscr{N}_{qc} \in \mathscr{C}(\mathscr{N}_q), s_{qc} = -\infty}
\max(s_q - s_q^{\min} - 1, 0) \right).
\end{equation}

(Note that $b$ can be calculated in a single $O(N)$ traversal of the tree.)

Lastly, case \textit{(3)} occurs when the reference scale is less than the
minimum query scale $s_q^{\min}$.  Define the minimum reference scale as
$s_r^{\min}$.  If $s_q^{\min} - s_r^{\min} > 0$, each query node may have up to
$s_q^{\min} - s_r^{\min}$ reference recursions of type \textit{(3)}, and there
are $N$ query nodes.  Thus, the total number of reference recursions of type
\textit{(3)} is bounded by $N \max(s_q^{\min} - s_r^{\min}, 0)$.

Combining these three cases, we get the following expression for an upper bound
$\theta$ on the total number of reference recursions:

\begin{equation}
\theta = \max(s_r^T - s_q^T, 0) + b + N (s_q^{\min} - s_r^{\min}).
\end{equation}

Now, we can express these in terms of more common constants which are related to
the aspect ratio of the data.  Define $\delta_q$ as the minimum nonzero pairwise
distance in $S_q$ and $\eta_q$ as the furthest pairwise distance in $S_q$ (and
define $\delta_r$ and $\eta_r$ accordingly for the set $S_r$).  Then, $s_r^T =
\lceil \log_2 \eta_r \rceil$, $s_q^T = \lceil \log_2 \eta_q$, $s_r^{\min} =
\lceil \log_2 \delta_r \rceil$, and $s_q^{\min} = \lceil \log_2 \delta_q
\rceil$.  This gives a more reasonable expression:

\begin{equation}
\theta = \max(\log_2(\eta_r / \eta_q), 0) + b + N \max(\log_2(\delta_q / \delta_r), 0).
\end{equation}

Returning to our original line of reasoning, we can combine these results to
give the result of the theorem:

\begin{equation}
O(c_r^4 \psi \chi | R^* | \theta) + O(c_r^4 \psi | R^* | N) \sim O(c_r^4 \psi
\chi |R^*| (N + \theta).
\end{equation}
\end{proof}
\vspace*{-0.8em}

A quick discussion of the numerous constants we have introduced and what they
mean.  In the monochromatic case, the terms of $\theta$ depending on $\eta_q$,
$\eta_r$, $\delta_q$, and $\delta_r$ disappear entirely.  However we are left
with the $b$ term, which is kind of a measure of imbalance of the tree.  The
best case is a case where the tree is perfectly balanced and there are no
`missing levels' where a child of a node is more than one level away from its
parent.  In practice, this quantity is probably much less than $N$ and should
not scale significantly with $N$ (those points which contribute to $b$ the most
will probably be outliers).  In the worst case, $b$ is bounded by the maximum
depth of the tree multiplied by the number of points, which is trivially $O(N
\log (\eta_q / \eta_r)$.  It is likely that a tighter bound of $O(N \log N)$
could be established.

There are two corner cases for the non-monochromatic case: all extra reference
recursions are of type \textit{(1)}, and all extra reference recursions are of
type \textit{(3)}.  The first case presents the best situation, because all of
the reference recursions are done before the first query node is recursed in.
Therefore the bulk of the reference recursion work is done by with the query
tree root as the query node; this is the best case.  The worst case is the
latter, where recursion in the reference tree cannot begin until reaching the
bottom of the query tree.  Thus, no pruning can occur until reaching the leaves
of the query tree, and the recursion is essentially equivalent to a single-tree
recursion for each query point plus the added overhead of the recursion of the
query tree with one reference node.  This second case degenerates to something
like $O(N \log N)$ or $O(N^2)$ time depending on how you look at it.

\subsection{Back to your regularly scheduled paper.}

This result holds for any dual-tree algorithm regardless of the problem. Hence,
%for any pairwise statistical problem, the runtime of the dual-tree algorithm (of
the runtime of any dual-tree algorithm
%that problem)
would be at least $O(N)$ using our bound, which matches the intuition that
answering $O(N)$ queries would take at least $O(N)$ time. For a particular
problem and data, if $c_r$, $|R^*|$, $\nu$, $\chi$ and
$\psi$ are bounded by constants independent of $N$ (for large enough $N$), then
the dual-tree algorithm for that problem has a runtime linear in $N$. Our
theoretical result separates out the problem-dependent and the
problem-independent elements of the runtime bound, which allows us to simply
plug in the problem-dependent bounds to get runtime bounds for any dual-tree
algorithm without requiring an analysis from scratch.
% This discussion already happened earlier... sort of.
%The expansion constant is related to a notion of intrinsic dimensionality of a
%dataset (as discussed earlier). A relatively small $c_r$ corresponds to
%a dataset with low intrinsic dimensionality, while a value of $c_r$ independent
%of $N$ implies that the intrinsic dimensionality of the data is independent of
%the number of points in the dataset. The latter is a reasonable assumption in
%many scenarios.
%Our bound has a linear dependence on the inverse constant of bichromaticity
%$\nu$; this results from the worst-case situation where the reference tree is
%recursed entirely after all query tree recursions
% this makes sense, given that $\nu$ represents the maximum number of times
%any set $R$ is descended for any query node $\mathscr{N}_q$ before
%$\mathscr{N}_q$ is recursed.
If $\nu \sim \kappa$, which is a reasonable assumption, this is a better result
than the results of Ram et.~al.  \cite{ram2009} who found a dependence of
$c_q^{4 \kappa}$ in their proofs.

The quantity $|R^*|$ bounds the amount of work that needs to be done for each
recursion. In the worst case, $|R^*|$ can be $N$. However,
dual-tree algorithms rely on branch-and-bound techniques to prune away
work (Lines \ref{alg:line:ct-dual-ref-score} and
\ref{alg:line:ct-dual-query-pruning} in Algorithm \ref{alg:cover-tree-dual}). A
small value of $|R^*|$ will imply that the algorithm is extremely successful in
pruning away work at Line \ref{alg:line:ct-dual-ref-score} in Algorithm
\ref{alg:cover-tree-dual}. An (upper) bound on $|R^*|$ (and the algorithm's
success in pruning work) will depend on the problem and the data.  As we will
show, bounding $|R^*|$ is often possible. % In addition, we will also show that
For many dual-tree algorithms, $\chi \sim \psi \sim O(1)$; often, cached
sufficient statistics~\cite{moore2000anchors} can enable $O(1)$ runtime
implementations of \texttt{BaseCase()} and \texttt{Score()}.

% Space constraints :(

%The runtime bound $\chi$ for the \texttt{BaseCase($p_q$, $p_r$)} function
%corresponds to the time required to process a pair of points. For example, in
%range search, \texttt{BaseCase()} corresponds to the
%time taken to compute $d(p_q, p_r)$ and determine if that falls into the desired
%range.  For general kernel summations, this corresponds to computing the
%pairwise kernel function between $p_q$ and $p_r$ and then summing it to the
%output for $p_q$.  In general, the runtime of \texttt{BaseCase()} is dominated
%by the pairwise function computation which is independent of $N$ and is usually
%$O(1)$. The \texttt{Score()} function is an operation on a pair of nodes and
%usually involves computing a bound which requires a pairwise function
%computation between the node centers (similar to the \texttt{BaseCase()}
%function except that it is for node centres) and certain cached node statistics
%(such as node radius in the metric space). Hence its runtime bound $\psi$ is
%usually a constant independent of $N$ or simply $O(1)$.
