\begin{thm}
\label{thm:ct-runtime}
%For any dual-tree algorithm using cover trees and the standard cover tree
%dual-tree pruning traversal (Algorithm \ref{alg:cover-tree-dual}) with reference
%set $S_r$ that has expansion constant $c_r$ and query set $S_q$, the runtime is
%bounded above by $O(c_r^4 | R^* | \nu \chi \psi N)$, where $ | R^* | $ is
%the maximum size of the reference set $R$ (line
%\ref{alg:line:ct-dual-input}) at any point in the dual-tree pruning traversal,
%$\chi$ is the longest possible running time of \texttt{BaseCase()}, and $\psi$
%is the longest possible running time of \texttt{Score()}.
Given a reference set $S_r$ of size $N$ with an expansion constant $c_r$ and a
set of queries $S_q$ of size $O(N)$, a standard cover tree based dual-tree
algorithm (Algorithm \ref{alg:cover-tree-dual}) takes $O(c_r^4 | R^* | \nu \chi
\psi N)$, where $ | R^* | $ is the maximum size of the reference set $R$ (line
\ref{alg:line:ct-dual-input}) during the dual-tree recursion, $\chi$ is
the maximum possible runtime of \texttt{BaseCase()}, and $\psi$ is the maximum
possible runtime of \texttt{Score()}.
\end{thm}

\begin{proof}
Consider a reference recursion (lines
\ref{alg:line:ct-dual-ref-recursion-start}--\ref{alg:line:ct-dual-ref-recursion-end}).
The work done in the base case loop from lines
\ref{alg:line:ct-dual-base-case-start}--\ref{alg:line:ct-dual-base-case-end} is
$O(\chi | R |)$.  Define $| R^* |$ to be the largest set $|R|$ for any scale
$s_r^{\max}$ and any query node $\mathscr{N}_q$ during the course of the
algorithm; then, it is true that $| R | \le | R^* |$.
%
Then, lines \ref{alg:line:ct-dual-ref-children} and
\ref{alg:line:ct-dual-ref-score} take $O(c_r^4 \psi | R |) \le O(c_r^4 \psi |
R^* |)$ time, because each reference node has $c_r^4$ children.  So, one
full reference recursion takes $O(c_r^4 \psi | R^* |)$ time.

Now, note that there are $O(N)$ nodes in $\mathscr{T}_q$.  Thus, line
\ref{alg:line:ct-dual-query-recursion} is visited $O(N)$ times.  Each of these
$O(N)$ visits to line \ref{alg:line:ct-dual-query-recursion} implies a
recursion, in which the reference set is descended up to $\nu$ times (lines
\ref{alg:line:ct-dual-ref-recursion-start}--\ref{alg:line:ct-dual-ref-recursion-end})
before the query node is descended or the algorithm terminates.  In addition,
each $O(N)$ recursion implies an $O(\psi |R|) \le O(\psi |R^*|)$ operation for
the calculation of $R'$ (line \ref{alg:line:ct-dual-query-pruning}).  Thus, the
full runtime of the algorithm is bounded as $O(c_r^4 |R^*| \nu \chi \psi N +
\psi |R^*| N) = O(c_r^4 |R^*| \nu \chi \psi N)$.
\end{proof}

This result holds for any dual-tree algorithm regardless of the problem. Hence,
%for any pairwise statistical problem, the runtime of the dual-tree algorithm (of
the runtime of any dual-tree algorithm
%that problem)
would be at least $O(N)$ using our bound, which matches the intuition that
answering $O(N)$ queries would take at least $O(N)$ time. For a particular
problem and data, if $c_r$, $|R^*|$, $\nu$, $\chi$ and
$\psi$ are bounded by constants independent of $N$ (for large enough $N$), then
the dual-tree algorithm for that problem has a runtime linear in $N$. Our
theoretical result separates out the problem-dependent and the
problem-independent elements of the runtime bound, which allows us to simply
plug in the problem-dependent bounds to get runtime bounds for any dual-tree
algorithm without requiring an analysis from scratch.
% This discussion already happened earlier... sort of.
%The expansion constant is related to a notion of intrinsic dimensionality of a
%dataset (as discussed earlier). A relatively small $c_r$ corresponds to
%a dataset with low intrinsic dimensionality, while a value of $c_r$ independent
%of $N$ implies that the intrinsic dimensionality of the data is independent of
%the number of points in the dataset. The latter is a reasonable assumption in
%many scenarios.
%Our bound has a linear dependence on the inverse constant of bichromaticity
%$\nu$; this results from the worst-case situation where the reference tree is
%recursed entirely after all query tree recursions
% this makes sense, given that $\nu$ represents the maximum number of times
%any set $R$ is descended for any query node $\mathscr{N}_q$ before
%$\mathscr{N}_q$ is recursed.
If $\nu \sim \kappa$, which is a reasonable assumption, this is a better result
than the results of Ram et.~al.  \cite{ram2009} who found a dependence of
$c_q^{4 \kappa}$ in their proofs.

The quantity $|R^*|$ bounds the amount of work that needs to be done for each
recursion. In the worst case, $|R^*|$ can be $N$. However,
dual-tree algorithms rely on branch-and-bound techniques to prune away
work (Lines \ref{alg:line:ct-dual-ref-score} and
\ref{alg:line:ct-dual-query-pruning} in Algorithm \ref{alg:cover-tree-dual}). A
small value of $|R^*|$ will imply that the algorithm is extremely successful in
pruning away work at Line \ref{alg:line:ct-dual-ref-score} in Algorithm
\ref{alg:cover-tree-dual}. An (upper) bound on $|R^*|$ (and the algorithm's
success in pruning work) will depend on the problem and the data.  As we will
show, bounding $|R^*|$ is often possible. % In addition, we will also show that
For many dual-tree algorithms, $\chi \sim \psi \sim O(1)$; often, cached
sufficient statistics~\cite{moore2000anchors} can enable $O(1)$ runtime
implementations of \texttt{BaseCase()} and \texttt{Score()}.

% Space constraints :(

%The runtime bound $\chi$ for the \texttt{BaseCase($p_q$, $p_r$)} function
%corresponds to the time required to process a pair of points. For example, in
%range search, \texttt{BaseCase()} corresponds to the
%time taken to compute $d(p_q, p_r)$ and determine if that falls into the desired
%range.  For general kernel summations, this corresponds to computing the
%pairwise kernel function between $p_q$ and $p_r$ and then summing it to the
%output for $p_q$.  In general, the runtime of \texttt{BaseCase()} is dominated
%by the pairwise function computation which is independent of $N$ and is usually
%$O(1)$. The \texttt{Score()} function is an operation on a pair of nodes and
%usually involves computing a bound which requires a pairwise function
%computation between the node centers (similar to the \texttt{BaseCase()}
%function except that it is for node centres) and certain cached node statistics
%(such as node radius in the metric space). Hence its runtime bound $\psi$ is
%usually a constant independent of $N$ or simply $O(1)$.
