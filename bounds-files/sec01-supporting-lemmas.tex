For simplicity, the algorithms considered in this paper will be presented in a
tree-independent context, as in \citet{curtin2013tree}, but the only type of
tree we will consider is the cover tree \citep{langford2006}, and the only type
of traversal we will consider is the cover tree pruning dual-tree traversal,
which we will describe below.

As we will be making heavy use of trees, we must establish notation \citep[taken
from][]{curtin2013tree}.  The notation we will be using is defined in Table
\ref{tab:notation}.

\begin{table}
{\small
\begin{center}
\begin{tabular}{|c|l|}
\hline
{\bf Symbol} & {\bf Description} \\ \hline
$\mathscr{N}$ & A tree node \\ \hline
$\mathscr{C}_i$ & Set of child nodes of $\mathscr{N}_i$ \\ \hline
$\mathscr{P}_i$ & Set of points held in $\mathscr{N}_i$ \\ \hline
$\mathscr{D}_i^n$ & Set of descendant nodes of $\mathscr{N}_i$ \\ \hline
$\mathscr{D}_i^p$ & Set of points contained in $\mathscr{N}_i$ and
$\mathscr{D}_i^n$ \\ \hline
$\mu_i$ & Center of $\mathscr{N}_i$ (for cover trees, $\mu_i = p_i$) \\ \hline
$\lambda_i$ & Furthest descendant distance \\ \hline
\end{tabular}
\end{center}
}
\caption{Notation for trees.  See \cite{curtin2013tree} for details.}
\label{tab:notation}
\end{table}

\subsection{The cover tree}

The cover tree is a leveled hierarchical data structure originally proposed for
the task of nearest neighbor search by \citet{langford2006}.  Each node
$\mathscr{N}_i$ in the cover tree is associated with a single point $p_i$.  An
adequate description is given in their work (we have adapted notation slightly):

\begin{quote}
A {\it cover tree} $\mathscr{T}$ on a dataset $S$ is a leveled tree where each
level is a ``cover'' for the level beneath it.  Each level is indexed by an
integer scale $s_i$ which decreases as the tree is descended.  Every {\it node}
in the tree is associated with a point in $S$.  Each {\it point} in $S$ may be
associated with multiple nodes in the tree; however, we require that any point
appears at most once in every level.  Let $C_{s_i}$ denote the set of points in
$S$ associated with the nodes at level $s_i$.  The cover tree obeys the
following invariants for all $s_i$:

\begin{itemize}
  \item {\em (Nesting)}. $C_{s_i} \subset C_{s_i - 1}$.  This implies that once a
point $p \in S$ appears in $C_{s_i}$ then {\it every} lower level in the tree
has a node associated with $p$.

  \item {\em (Covering tree)}. For every $p_i \in C_{s_i - 1}$, there exists a
$p_j \in C_{s_i}$ such that $d(p_i, p_j) < 2^{s_i}$ and the node in level $s_i$
associated with $p_j$ is a parent of the node in level $s_i - 1$ associated with
$p_i$.

  \item {\em (Separation)}.  For all distinct $p_i, p_j \in C_{s_i}$, $d(p_i,
p_j) > 2^{s_i}$.
\end{itemize}
\end{quote}

A consequence of this definition is if there exists a node $\mathscr{N}_i$,
containing the point $p_i$ at some scale $s_i$, then there will also exist a
self-child node $\mathscr{N}_{ic}$ containing the point $p_i$ at scale $s_i - 1$
which is a child of $\mathscr{N}_i$.  In addition, every descendant point of the
node $\mathscr{N}_i$ is contained within a ball of radius $2^{s_i + 1}$ centered
at the point $p_i$; therefore, $\lambda_i = 2^{s_i + 1}$ and $\mu_i = p_i$
(Table \ref{tab:notation}).

Note that the cover tree may be interpreted as an infinite-leveled tree, with
$C_{\infty}$ containing only the root point, $C_{-\infty} = S$, and all levels
between defined according to the definitions above.  \citep{langford2006} find
this representation (which they call the {\it implicit} representation) easier
for description of their algorithms and some of their proofs.  But naturally,
this is not suitable for implementation; hence, there is an {\it explicit}
representation in which all nodes that have only a self-child are coalesced.
Figure \ref{fig:implicit-explicit} shows an example cover tree and highlights
implicit and explicit nodes.

In our work here, we consider only the explicit representation of a cover tree,
and do not concern ourselves with the construction of the tree\footnote{A batch
construction algorithm is given by \citet{langford2006}, called
\texttt{Construct}.}.

\subsection{Expansion constant}

The explicit representation of a cover tree has a number of useful theoretical
properties based on the expansion constant \citep{karger2002finding}; we restate
its definition below.

\begin{defn}
\label{def:int_dim}
Let $B_S(p, \Delta)$ be the set of points in $S$ within a closed ball of radius
$\Delta$ around some $p \in S$ with respect to a metric $d$:
%
$B_S(p, \Delta) = \{ r \in S \colon d(p, r) \leq \Delta \}$.
%
Then, the {\bf expansion constant} of $S$ with respect to the metric $d$ is the
smallest $c \ge 2$ such that

\begin{equation}
| B_S(p, 2 \Delta) | \le c | B_S(p, \Delta) |\ \forall\ p \in S,\
\forall\ \Delta > 0.
\end{equation}

\end{defn}

The expansion constant is used heavily in the cover tree literature.  It is,
in some sense, a notion of instrinic dimensionality, and previous work has shown
that there are many scenarios where $c$ is independent of the number of points
in the dataset \citep{karger2002finding, langford2006,
krauthgamer2004navigating, ram2009}.  Note also that if points in $S \subset
\mathcal{H}$ are being drawn according to a stationary distribution $f(x)$, then
$c$ will converge to some finite value $c_f$ as $|S| \to \infty$.  To see this,
define $c_f$ as a generalization of the expansion constant for distributions.
$c_f \ge 2$ is the smallest value such that

\begin{equation}
\int_{\mathcal{B}_{\mathcal{H}}(p, 2 \Delta)} f(x) dx \le c_{f}
\int_{\mathcal{B}_{\mathcal{H}}(p, \Delta)} f(x)
dx
\end{equation}

\noindent for all $p \in \mathcal{H}$ and $\Delta > 0$ such that
$\int_{\mathcal{B}_{\mathcal{H}}(p, \Delta)} f(x) dx > 0$, and with
$\mathcal{B}_{\mathcal{H}}(p, \Delta)$ defined as the closed ball of radius
$\Delta$ in the space $\mathcal{H}$.

Take $f(x)$ as a uniform spherical distribution in $\mathcal{R}^d$: for any $|x|
\le 1$, $f(x)$ is a constant; for $|x| > 1$, $f(x) = 0$.  It is easy to see that
$c_f$ in this situation is $2^d$.  To demonstrate this convergence, Table
\ref{tab:ec_scaling} shows results for 10 trials of empirically calculated
expansion constants for datasets drawn from this distribution, for varying $d$.

\begin{table}[htb]
\begin{center}
\begin{tabular}{|l|c|c|c|c|}
\hline
$|S|$ & $d = 2$ & $d = 5$ & $d = 10$ & $d = 20$ \\
\hline
10      & 6     & 10    & 10    &         \\
31      & 12    & 31    & 31    &         \\
100     & 13    & 64    & 100   &         \\
316     & 23    & 96    & 286   &         \\
1000    & 26    & 126   & 673   &         \\
3162    & 29    & 189   & 1348  &         \\
10000   & 30    & 259   & 2646  &         \\
31622   & 39    & 309   & 3858  &         \\
100000  & 40    & 360   & 7162  &         \\
316227  & 46    & 385   & 8297  &         \\
1000000 &       &       &       &         \\
$\infty$ ($c_f$)   & 4     & 32    & 1024  & 1048576 \\
\hline
\end{tabular}
\end{center}
\caption{Calculated $c$ for uniform hypersphere distribution as
$|S|$ scales.}
\label{tab:ec_scaling}
\end{table}

We are somewhat limited in the sizes of $|S|$ we can scale to, because
calculating the expansion constant of a dataset is a computationally intensive
task.  Nonetheless, we can see that the expansion constant moves towards the
final value of $c_f$, though potentially from above ({\it The results are not
done yet.  This paragraph may need rewriting.  But I strongly suspect the
results will show what I've said.}).  We postulate that the behavior of more
complex real-world datasets is similar.

There are some other important observations about the behavior of $c$.  Adding a
single point to $S$ may increase $c$ arbitrarily: consider a set $S$ distributed
entirely on the surface of a unit hypersphere.  If one adds a single point at
the origin, producing the set $S'$, then $c$ explodes to $|S'|$ whereas before
it may have been much smaller than $|S|$.  Adding a single point may also
decrease $c$ significantly.  Suppose one adds a point arbitrarily close to the
origin to $S'$; now, the expansion constant will be $|S'| / 2$.  So, although we
can bound the behavior of $c$ as $|S| \to \infty$ for $S$ from a stationary
distribution, we are not able to easily say much about its convergence behavior.

The expansion constant can be used to show a few useful bounds on various
properties of the cover tree; we restate these results below, given some cover
tree built on a dataset $S$ with expansion constant $c$ and $|S| = N$:

\begin{itemize}
  \item {\bf Width bound:} no cover tree node has more than $c^4$ children
(Lemma 4.1, \cite{langford2006}).

  \item {\bf Depth bound:} the maximum depth of any node is $O(c^2 \log N)$
(Lemma 4.3, \cite{langford2006}).

  \item {\bf Space bound:} a cover tree has $O(N)$ nodes (Theorem 1,
\cite{langford2006}).
\end{itemize}

%\begin{lemma}
%\label{lem:width}
%(Lemma 4.1, \cite{langford2006}) The number of children of any cover tree node $\mathscr{N}_i$ is bounded by
%$c^4$, where $c$ is the expansion constant of the dataset the cover tree is
%built on, as defined in Definition \ref{def:int_dim}.
%\end{lemma}

%\begin{lemma}
%\label{lem:depth}
%(Lemma 4.3, \cite{langford2006}) The maximum depth of any point $p_r$ in a cover
%tree $\mathscr{T}_r$ is $O(c^2 \log N)$, where $N$ is the number of points in
%the dataset that $\mathscr{T}_r$ is built on.
%\end{lemma}

Lastly, we introduce a convenience lemma of our own which is a generalization of
the packing arguments used by \citet{langford2006}.  This is a more flexible
version of their argument.

\begin{lemma}
Consider a dataset $S$ with expansion constant $c$ and a subset $C \subseteq S$
such that every point in $C$ is separated by $\delta$.  Then, given any query
point $p \not\in S$ and some radius $\rho \delta$,

\begin{equation}
| B_S(p, \rho \delta) \cap C | \le c^{2 + \log_2 \rho}.
\end{equation}
\label{lem:packing}
\end{lemma}

\begin{proof}
This is based on the packing argument from Lemma 4.1 in \cite{langford2006}.
Observe that $B_S(p, \rho \delta) \subseteq B_S(p_i, 2 \rho \delta)$ for any
$p_i \in S$, and that $| B_S(p, 2 \rho \delta) | = c^{2 + \log_2 \rho} |
B_S(p, \delta / 2) |$.  Because each point in $C$ is separated by $\delta$, the
number of points in $B_S(p, \rho \delta) \cap C$ is
bounded by the number of disjoint balls of radius $\delta / 2$ that can be
packed into $B_S(p, \rho \delta)$.  In the worst case, this packing is
perfect, and

\begin{equation}
|B_S(p, \rho \delta)| \le \frac{|B_S(p_i, 2 \rho \delta)|}{|B_S(p_i, \delta
/ 2)|} \le c^{2 + \log_2 \rho}.
\end{equation}
\end{proof}
